\chapter{Ataques de dicionário}\label{1_ataques_servidores}

Existem inúmeras formas de ataque empregadas contra servidores e computadores convencionais e muitas delas são descritas pelo \textit{SANS Institute} em suas publicações periódicas \cite{Top20Sans}.

Muitas das ameaças levantadas, são recorrentes exclusivamente aos sistemas \textit{Desktop}, muitas vezes suscetíveis a variadas combinações de \textit{worms}, \textit{malwares} e \textit{vírus}, o que afeta somente aos usuários finais dos sistemas. Também existem aquelas empregadas quase que exclusivamente contra os servidores, tais como ataques \textit{Denial of Service} em rede, \textit{Buffers Overflows} em rede, XSS e Ataques de dicionário por força bruta sempre estiveram listados entre os expressivos \cite{Top10Sans} \cite{Top20Sans}.

Quando servidores são afetados por alguma forma de ataque, os resultados são muito mais drásticos, quando comparados aos causados contra computadores \textit{Desktops}. Como sua finalidade é prover um ou mais serviços a uma variada gama de usuários e até mesmo a outros sistemas, quando os servidores são comprometidos os prejuízos podem ser, em muitas das vezes, incalculáveis. Devido sua importância, o tema escolhido para este trabalho, foram os Ataques de dicionário por força bruta lançados contra servidores da família \textit{Unix}, restringindo-se aos que possuam o serviço \textit{SSH} \footnote{\textit{SSH} é um acrônimo para \textit{\textbf{S}ecure \textbf{SH}ell}. Sua variante mais utilizada atualmente, é o \textit{OpenSSH}, sendo padrão na grande maioria das distribuições baseadas no \textit{Kernel} do \textit{GNU/Linux}. \textit{Website} do projeto disponível em: \url{http://www.openssh.com/}} rodando como \textit{daemon} \footnote{\textit{Daemon} é uma expressão frequentemente usada para referir-se a processos que rodam em \textit{background}, mas que proveêm algum nível de interação com os clientes que utilizem seus protocolos de comunicação em rede, sendo para tanto, processos que são a base do funcionamento de tecnologias em servidores.}.

Caso o atacante deseje tomar o controle das operações em servidores, seu sucesso pode depender da exploração de algum serviço que seja capaz de disponibilizar acesso à linha de comando, \textit{Shell} \footnote{O \textit{Shell} faz parte da \textit{CLI}, \textit{Command Line Interface} do Sistema Operacional, sendo capaz de executar ações sem que seja necessário o modo gráfico, de maneira mais rápida e mais eficiente, com uso de um conjunto determinado de comandos e ferramentas padrão.} do Sistema Operacional implantado. Uma vez que o acesso tenha sido garantido, deve-se avaliar os privilégios obtidos em sua investida e, caso estes sejam escassos, o passo seguinte para a tomada do sistema, certamente será a utilização de técnicas que possam escalar seus privilégios até às contas administrativas. Isso é possível, através da exploração de vulnerabilidades conhecidas em versões desatualizadas  de programas e serviços que por ventura estejam em execução no sistema alvo, ou mesmo que estejam meramente disponíveis para utilização, mas que possam ainda sim, trazer riscos. \cite{PrivilegeEscalation}

Historicamente, têm-se utilizado o serviço \textit{SSH} para acesso remoto e controle de servidores. Seu uso foi favorecido em comparação ao antigo \textit{Telnet} \footnote{\textit{Telnet} é um acrônimo para \textit{\textbf{Tel}etype \textbf{Net}work}, trata-se de um protocolo que permite a comunicação interativa bidirecional em rede, utilizado em conexões cliente/servidor. Os dados trocados não trafegam criptografados, sendo hoje considerada uma maneira insegura de se acessar informações em servidores remotos.}, por oferecer um canal seguro entre o modelo cliente/servidor, onde as informações trafegam criptografadas pela rede. No entanto, o serviço \textit{SSH}, assim como todas as tecnologias, não é infalível à falhas de segurança, tendo como uma das principais ameaças, os ataques de dicionários por força bruta.

Existem inúmeras ferramentas que lançam, automaticamente, ataques de dicionário por força bruta via \textit{SSH}, como a \textit{SSH Brute Forcer} \footnote{\textit{SSH Brute Forcer} é uma ferramenta que lança, automaticamente, ataques por força bruta via em servidores, levando em consideração um dicionário de palavras contendo as ocorrências mais comuns de usuários/senhas. Escrita na linguagem de programação \textit{scripting Python}, pode ser encontrada em \url{http://packetstormsecurity.org/Crackers/sshbrute.py.txt}. Para seu funcionamento, é necessária a instalação do módulo \textit{Python Pexpect}, disponível em \url{http://pexpect.sourceforge.net/}}. Um exemplo comum de saída padrão, gerado pela sua utilização contra um alvo arbitrário e sucetível a falhas, seria:

\begin{verbatim}
bode@bodacious:~/pacotes/seguranca$ ./sshbrute 127.0.0.1 test wordlist

           d3hydr8:darkc0de.com sshBrute v1.0
        ----------------------------------------

[+] Loaded: 7 words
[+] Server: 127.0.0.1
[+] User: test
[+] BruteForcing...

Trying: test
Trying: aaaa
Trying: 123
Trying: 123456

uname -a
Linux bodacious 2.6.29.6-smp #2 SMP Mon Aug 17 00:52:54 CDT 2009 i686
AMD Athlon(TM) XP 2200+ AuthenticAMD GNU/Linux

        [!] Login Success: test 123456
\end{verbatim}

Por meio deste exemplo simples, pôde ser observado o funcionamento básico de um típico ataque de dicionário para SSH, que iterativamente testa sob um usuário específico, várias senhas contidas em um arquivo específico, chamado \textit{Wordlist}, que contém as recorrências mais comuns destas em sistemas vulneráveis. Este ataque foi lançado sob o computador local, que propositalmente continha o usuário \textit{test}, cuja senha fora identificada como \textit{123456} dentre as demais, contidas na \textit{Wordlist}.

Com pouco esforço, pode-se automatizar ainda o processo envolvido nos ataques como o que o \textit{SSH Brute Forcer} realiza. É possível, por exemplo, cruzar os dados de uma só \textit{Wordlist} com os mesmos iterados, tendo assim, uma combinação completa entre usuários e senhas. Tem como vantagem, o teste exaustivo de todas as combinações, mas apresenta como principal desvantagem, um grande aumento no tempo de execução do processo, por ser definido em termos de uma função de ordem quadrática, denotada em notação assintótica, por \textit{O(\begin{math}n^2\end{math})}, uma vez que trata-se do produto cartesiano dos elementos contidos na \textit{Wordlist} consigo mesmo.

\section{Ataques de dicionário para SSH}

Um ataque de dicionário, é assim classificado devido à utilização de um arquivo contendo inúmeras combinações de usuários e senhas para fins específicos, tal arquivo é comumente referido como \textit{Wordlist}. No caso do serviço \textit{SSH}, utilizam-se métodos automatizados para se obter o sucesso em ataques desse tipo.

Ferramentas como \textit{ssh-dictattack} \footnote{O código-fonte da ferramenta \textit{ssh-dictattack} está disponível em \url{http://ircsex.de/download/utils/ssh-dictattack.c}} e \textit{SSH Brute Forcer}, abordado anteriormente, fazem uso de dicionários, lançando na rede, sucessivas tentativas de se estabelecer conexões com servidores que dispõe de um \textit{daemon} \textit{SSH} rodando.

Os servidores são escolhidos através de \textit{scans} previamente realizados nas redes \cite{NetworkBasedDictionaryAttackDetection}. Uma vez identificado o serviço \textit{SSH} rodando em um servidor, este é imediatamente agregado à listas que contém os possíveis alvos. Também existe o caso onde o atacante escolhe cuidadosamente suas vítimas, existindo assim, interesse específico nas mesmas, em geral relacionado ao alto poder computacional de determinadas instituições. É também muito comum, que computadores comprometidos sejam integrados às redes zumbis e sejam controlados remotamente, de modo que lancem diversos tipos de \textit{scans} em rede na busca de máquinas também comprometidas e/ou que rodem serviços interessantes aos atacantes, como é o caso do \textit{SSH}.

Para que os dicionários sejam gerados, contendo em milhares de entradas de usuários e senhas, utiliza-se ferramentas intermediárias e totalmente automatizadas, como é o caso da chamada \textit{CeWL} \footnote{\textit{CeWL} é um acrônimo para \textit{Custom Word List generator}. É uma ferramenta para a geração automática de \textit{Wordlists}, que podem ser utilizadas em diversos ataques baseados em dicionários. O site do projeto é disponível em: \url{http://www.digininja.org/projects/cewl.php}}.

Uma vez que o serviço \textit{SSH} esteja sofrendo com ataques de dicionário, são gravadas as informações relacionadas às fontes dos ataques, mostrando em logs no servidor \footnote{O referido log é, em geral, o \textit{var/log/messages}, padrão em muitos dos sistemas baseados na família \textit{Unix}.}, os endereços \textit{IP} responsáveis pelos ataques e a data em que cada incidente ocorreu.

Em sistemas da família Unix, tais registros de log são armazenados, por padrão, no arquivo \textit{/var/log/messages}. Uma maneira nativa, de se obter informações sobre as investidas contra o sistema em questão, é através da interpretação dos dados armazenados, estabelecendo-se filtros baseados em expressões regulares. Desse modo, com a formação de uma expressão satisfatória, os dados filtrados correspondem exatamente às tentativas de entrada no sistema e, revelam na maioria das vezes, a presença de usuários não-cadastrados, mas encontrados nos dicionários dos atacantes.

\section{Análise de logs SSH}

É de grande valia, para os administradores de sistemas, que as informações referentes aos ataques baseados em dicionário para \textit{SSH} possam ser visualizadas de maneira rápida e com o menor esforço possível. Desse modo, o monitoramento torna-se mais ágil e mediante a sucessivas tentativas de entrada não-autorizada no sistema, pode-se estabelecer políticas para que o \textit{firewall} barre futuras tentativas provenientes de indivíduos específicos, antes delatados por seus endereços de \textit{IP}, registrados no \textit{log} referido.

Foi escolhido como tema deste estágio, a criação de uma ferramenta capaz de fornecer aos administradores de sistemas, uma maneira simples e de fácil acesso aos ataques com base nos dados gerados pelo serviço \textit{SSH}, armazenados em seus servidores, formatados como arquivos de log, optou-se também, pela possibilidade de se realizar upload de logs para o sistema \textit{Web}, uma vez que os arquivos podem estar armazenados de maneira dispersa. Visando facilitar o acesso à ferramenta e suas funcionalidades, foi escolhido um conjunto de tecnologias voltadas para a \textit{Web} em sua implementação e implantação. Uma vez disponível em formato \textit{Hypertexto}, as consultas  podem ser realizadas a partir de uma gama enorme de meios que suportem navegação, variando desde computadores pessoais até dispositivos \textit{mobile}.

A ferramenta foi batizada como \textit{Crickets' little leg}, tendo como principais características, a facilidade de utilização e a simplicidade empregada em sua interface. Também foram levantados aspectos referentes à portabilidade da mesma, sendo assim, este requisito foi levado adiante até mesmo na escolha das tecnologias empregadas para o seu desenvolvimento, abrangendo com sucesso, a maioria dos ambientes em que encontram-se servidores baseados na família \textit{Unix}, o que traz como consequência positiva, um processo de implantação descomplicado e extremamente viável.