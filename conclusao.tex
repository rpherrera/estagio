\chapter*{Conclusão}

Foi apresentada neste trabalho a ferramenta \textit{Crickets' little leg}, concebida no intuito de auxiliar os administradores de sistemas baseados em famílias \textit{Unix} a identificar informações sobre ataques de dicionários lançados contra o serviço \textit{SSH}, a partir dos \textit{logs} registrados. Através de uma interface \textit{Web} simples e intuitiva, as seguintes informações são recuperadas:

\begin{singlespace}
    \begin{itemize}
        \item Os endereços de \textit{IP} dos responsáveis pelos ataques
        \item Os horários em que os ataques ocorreram
        \item Os países de onde foram lançados os ataques
    \end{itemize}
\end{singlespace}

Além do processo de extração de informações, a ferramenta permite a geração automática de regras para \textit{Iptables}, que uma vez aplicadas nos servidores, bloqueiam efetivamente os pacotes de rede provenientes dos enderços de \textit{IP} considerados suspeitos.

A ferramenta foi desenvolvida no intuito de atender a demanda da maioria dos ambientes onde se pode hospedadar aplicações \textit{Web}, não exigindo assim, requisitos adicionais, que por ventura possam inviabilizar sua implantação.

Como trabalhos futuros, a ferramenta poderá:

\begin{itemize}
    \item Agregar análises relacionadas a diferentes formas de ataque, como por exemplo, a identificação de ataques \textit{DoS} específicos, bastando que sua configuração seja modificada e seus módulos sejam programados, se integrando facilmente ao modelo \textit{MVC} empregado na arquitetura do projeto.
    \item Oferecer suporte um banco de dados de localidades ainda mais específicas, revelando não somente os países cujos ataques foram lançados, mas também, as cidades de onde estes partiram.
\end{itemize}
