\chapter*{Introdução}

Desde o ano 2000, o \textit{SANS Institute}, em conjunto com o \textit{NIPC}, uma divisão pertencente ao \textit{FBI}, vem compilando uma lista com as ameaças mais recorrentes aos computadores interconectados em rede, especialmente à \textit{Internet}. A primeira lista fora classificada como \textit{``As dez vulnerabilidades de Internet mais críticas''} \cite{Top10Sans}, sendo expandida durante os anos seguintes para \textit{``Os vinte maiores problemas, ameaças e riscos da Internet''} \cite{Top20Sans}. Os esforços por parte de seus responsáveis, vêm sendo direcionados até os dias de hoje no intuito cobrir os pontos mais frágeis e comuns aos ambientes interconectados em rede.

Dentre tais ameaças, podem ser ressaltadas como mais marcantes, as que se relacionam aos seguintes tópicos:

\begin{itemize}
    \item Sistemas operacionais podem apresentar algumas falhas que uma vez exploradas, podem acarretar na distribuição massiva de \textit{worms} em sistemas conectados à Internet.
    \item Falhas são frequentemente encontradas em aplicações-cliente, tais como navegadores, ferramentas pertencentes à suites \textit{Office} e reprodutores de mídia. Uma vez exploradas, podem levar à corrupção do sistema hospedeiro.
    \item Funcionários de corporações e instituições com acesso à Internet estão expostos à diversos riscos oferecidos por páginas da \textit{WWW}, \sigla{WWW}{World Wide Web} maliciosamente manipuladas, sendo a porta de entrada para \textit{malwares} nas redes privadas.
    \item Falhas encontradas tanto em sistemas \textit{Web open-source} quanto em sistemas \textit{Web} proprietários, podem conduzir à transformação de sites confiáveis em versões manipuladas maliciosamente pelos atacantes.
    \item Muitas configurações-padrão de sistemas operacionais e serviços permitem que tais características sejam exploradas, sendo comprometidos via ataques de força-bruta baseados em dicionários.
    \item Os interessados em atacar as instituições, obtém informações sensíveis via métodos cada vez mais inovadores, sendo uma tarefa essencial, que se estabeleça um controle mais aguçado de como as informações deixam os limites das organizações.
\end{itemize}

Neste trabalho, será abordada uma forma de ameaça específica relacionada aos ataques de força-bruta, são os ataques de dicionário contra servidores baseados na família \textit{Unix} e que possuem o serviço \textit{SSH} rodando.

Tais ataques são promovidos com o auxílio de ferramentas automatizadas, que utilizam um conjunto de combinações de usuários e senhas comuns, catalogados em arquivos chamados  \textit{dicionários}. Quando reincidentes, essas tentativas são armazenadas em forma de \textit{logs} nos servidores.

Observando a importância desta forma de ataque, foi proposto como estágio, o desenvolvimento de uma ferramenta \textit{Web}, capaz de gerar relatórios que identifiquem as tentativas de acesso que foram negadas ao serviço \textit{SSH}. A partir da análise dos dados identificados pela ferramenta, pode-se realizar uma melhoria nos filtros de \textit{firewall} presentes nos servidores, através da elaboração de regras específicas de \textit{Iptables}.

Este trabalho cobre os fundamentos teóricos e práticos sobre a problemática envolvida, assim como a implementação de uma ferramenta \textit{Web} que busca por tentativas não autorizadas de acesso ao serviço \textit{SSH} em servidores, a partir da análise de seus logs gerados, auxiliando a resolução deste problemas através da geração automática de um conjunto de regras para \textit{Iptables}, que podem ser utilizadas para bloquear futuras tentativas de acesso por parte dos atacantes. Os capítulos são divididos da seguinte maneira:

\begin{itemize}
    \item \textit{Capítulo 1}: Descreve os ataques de dicionário, empregados contra os servidores da família \textit{Unix} e que rodam o serviço \textit{SSH}. Exemplifica este problema a partir do lançamento de um ataque bem sucedido contra o \textit{host} local, preparado propositalmente para o teste. Introduz a idéia de se ter uma ferramenta capaz de analisar os logs gerados a partir dessa modalidade de ataques.
    \item \textit{Capítulo 2}: Apresenta a ferramenta que fora idealizada no Capítulo \ref{1_ataques_servidores}. Mostra um levantamento sobre a popularidade das tecnologias utilizadas em servidores \textit{Web}, justificando a influencia que exerceram sobre o processo de criação da ferramenta. Exemplifica sua utilização e aborda suas características.
    \item \textit{Capítulo 3}: Mostra a engenharia empregada no desenvolvimento do software e aborda aspectos sobre a metodologia de desenvolvimento utilizada.
\end{itemize}

\sigla{API}{Application Programming Interface}
\sigla{AUP}{Agile Unified Process}
\sigla{CLI}{Command Line Interface}
\sigla{CeWL}{Custom Word List generator}
\sigla{CSS}{Cascading Style Sheet}
\sigla{DoS}{Denial of Service}
\sigla{FBI}{Federal Bureau of Investigation}
\sigla{GNU}{GNU Not Unix}
\sigla{LGPL}{Lesser General Public License}
\sigla{HTML}{Hypertext Markup Language}
\sigla{HTTP}{Hypertext Transfer Protocol}
\sigla{IP}{Internet Protocol}
\sigla{MVC}{Model View Controller}
\sigla{NIPC}{National Infrastructure Protection Center}
\sigla{PHP}{PHP: Hypertext Preprocessor}
\sigla{RUP}{Rational Unified Process}
\sigla{SANS}{SysAdmin, Audit, Network, Security}
\sigla{SSH}{Secure SHell}
\sigla{URL}{Uniform Resource Locator}
\sigla{XML}{Extensible Markup Language}
\sigla{XSS}{Cross Site Script}